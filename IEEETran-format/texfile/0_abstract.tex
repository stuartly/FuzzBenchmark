\begin{abstract}
American Fuzzy Lop (AFL) is one of the most effective fuzzing tools to explore vulnerabilities in commercial off-the-shelf software. Existing works improve AFL's abilities to maximize code coverage of tested programs. However, none of them could explore the entire space of real-world applications in practice exhaustively, especially when testing resources (time or computation) are limited. Thus, the strategy of distributing fuzzing energy is essential. Existing energy distribution strategies of AFL and its variants have two limitations. (1) They focus on increasing coverage, but lack guidance to direct the fuzzing tool to approach code regions that are more likely to be vulnerable. (2) Although the mutation number of a seed can be adjusted, existing works randomly select mutators and therefore lack insights regarding which kind of mutators are more helpful at that particular stage.

We leverage these two new insights to improve AFL's fuzzing energy distribution in a principled way. We direct the fuzzer to strengthen fuzzing toward regions that have higher chance to contain vulnerabilities based on static semantic metrics of the target program. Furthermore, granularity-aware scheduling of energy distribution is proposed, which dynamically assigns ratio to different mutation operators based on their current performance. We implemented these improvements as an extension to AFL. Large-scale experimental evaluations showed the effectiveness of each improvement and performance of integration. The proposed tool has helped us find five unknown bugs and identify one new CVE in Libtasn1-4.13.

\end{abstract}