
\section{Related Work}\label{relatedwork}
% Researchers have done a lot of work to improve AFL's capabilities, which could be categorized into three directions:

We review related work that improves AFL's capabilities from three dimensions.

(1) \textbf{Effectiveness:} Effectiveness concerns about meta-abilities of fuzzers to bypass obstacles and trigger vulnerabilities. Generally speaking, there are three approaches to improving effectiveness. %as (1) feedback accuracy \cite{gancollafl} and granularity \cite{li2017steelix}; (2) smart mutate strategies ( i.e. where and what to mutate) \cite{zhang2018ptfuzz}  \cite{chen2013angora};  (3) sensitive to security violation \cite{serebryany2012addresssanitizer} \cite{stepanov2015memorysanitizer} and so on.
% as follows:

%\textbf{Improvement of  effectiveness}, detailed improvement could be concluded as some aspects as follows: 

\begin{itemize}

\item  \textit{Feedback granularity and accuracy:} Steelix \cite{li2017steelix} instruments AFL to collect mutate progress for magic bytes compassion, and continue to mutate on input that has progressed for each bit. CollAFL \cite{gancollafl} demonstrates that inaccuracy of feedback (i.e., hash collision issue) in AFL would limit the effectiveness of discovering path. The authors design an algorithm to resolve the hash collision problem, and improve the edge coverage accuracy with a low-overhead instrumentation scheme.  

\item \textit{Smart mutation strategy:} The mutate strategy answers the questions about where to mutate and what to mutate. To answer the former, Lin et al.\cite{lin2008convicting} propose a solution to identify raw bytes to mutate using static data lineage analysis. A deep neural network solution is proposed in \cite{rajpal2017not} to predicate which bytes to mutate. To answer the latter, Vuzzer \cite{rawat2017vuzzer} uses dynamic analysis to infer exceptional values (e.g., magic numbers to use for mutating.) 

\item \textit{Sensitive to security violation:} Fuzzers usually use program crashes as an indicator of vulnerabilities, because they are easy to be detected even without instrumentation. However, programs do not always crash when a vulnerability is triggered, e.g., when a padding byte following an array is overwritten. Researchers have proposed several solutions to detect various kinds of security violations. For example, the widely used AddressSanitizer\cite{serebryany2012addresssanitizer} and MemorySanitizer\cite{stepanov2015memorysanitizer} could detect buffer overflow and use-after-free vulnerabilities. There are many other sanitizers available, including UBSan\cite{lee2015type}, DataFlowsanitizer \cite{DataFlowSanitizer}, ThreadSanitizer \cite{serebryany2009threadsanitizer} and so on.
\end{itemize}



(2) \textbf{Efficiency:} Efficiency concerns about improving code coverage in order to improve the probability of trigger more vulnerabilities. Previous efforts were conducted following three approaches.% like (1) providing high quality and diversity of intial seeds \cite{wang2017skyfire} \cite{godefroid2017learn} \cite{nichols2017faster} \cite{lv2018smartseed}; (2) Improving the execution speed by prioritize faster and smaller seed, using new primitives \cite{xu2017designing} and utilizing system fork mechanism and hardware features (e.g Intel-PT) \cite{schumilo2017kafl} \cite{zhang2018ptfuzz}; (3) Balance the fuzzing energy distribution like low frequnce path, untouched path deserve more enery \cite{bohme2016aflfast}  \cite{gancollafl}.
% as follows:


%\textbf{Improvement of efficiency}, detailed improvement could be concluded as some aspects as follows: 

\begin{itemize}

\item  \textit{Quality and diversity of initial seeds:} Skyfire \cite{wang2017skyfire} learns a probabilistic context-sensitive grammar from abundant inputs to guide seed generation. Learn \& input \cite{godefroid2017learn} utilizes recurrent neural network(RNN) solution to generate valid seed files and could help generate inputs to pass format checks. Nicole Nichols et al.~\cite{lv2018smartseed} proposed a generative adversarial network(GAN) solution to argument the seed pool with extra seeds.
 % showing another promising solution.

\item  \textit{Execution speed:} AFL utilizes fork mechanism of Linux to accelerate the execution speed. It further uses forkserver mode and persistent mode to reduce the overhead of the fork operations. Besides, AFL prioritizes seeds that are executed faster, and thus it is likely that more test cases could be tested in a given time. Moreover, AFL also supports parallel mode, which enables multiple fuzzer instances to collaborate with each other. kAFL\cite{schumilo2017kafl} and PTfuzzer\cite{zhang2018ptfuzz} use hardware features (i.e., Intel PT) to accelerate the execution speed. Wen Xu et al. proposes several new primitives \cite{xu2017designing} which speed up AFL by 6.1x to 28.9x. 
 
\item \textit{Balance of fuzzing energy distribution:} AFLFast \cite{bohme2016aflfast} prioritizes seeds that were exercising less-frequent paths. Thus it is more likely that cold paths could be tested thoroughly and less energy will be wasted on hot paths.
% in order to balance the energy on the cold and hot paths. 
FairFuzz \cite{fairfuzz} spends more energy on less reachable branches. CollAFL\cite{gancollafl} prioritizes seeds that hit more untouched neighbors to improve the possibility to cover more new paths.
\end{itemize}


(3)  \textbf{Guidance:} The guidance criteria directs a fuzzer to focus on a specific goal. %Typical,  AFLGo \cite{bohme2017aflgo} directed fuzzing to reach some specific location as soon as possible.
%\textbf{Improvement of directiveness} directed grey box fuzzing be modeled as an object optimizing problem, detailed improvement work are including:
Along this direction, several works have been proposed.
% Detailed improvement works are including:
\begin{itemize}
\item \textit{Directing for specific locations:} AFLGo \cite{bohme2017aflgo} uses distance metrics to direct fuzzing to trigger and reproduce vulnerabilities in the specific location by giving the target locations at priority.

\item \textit{Directing for specific kinds bugs:} SlowFuzz \cite{petsios2017slowfuzz} prioritizes seeds that use more resources (e.g., CPU, memory, and energy), and try to increase the probability of triggering algorithmic complexity vulnerabilities. 
\end{itemize}
